\documentclass{article}

\usepackage{fullpage}
\usepackage[utf8]{inputenc}
\usepackage{csquotes}
\usepackage{hyperref}

\begin{document}
\title{Documentation of Polyite's Configuration Parameters}
\maketitle
Polyite reads its configuration from a file. The file contains pairs of the form
\texttt{key = value}, one pair per line. In the following, we document the available configuration options.

\section{General}
\begin{description}
  \item[logToFile] Specifies, whether to log to a file (true/ false)
  \item[logFile] log file. \textsc{Polyite} versions the log files.
  \item[islComputeout] Maximum number of computation steps, that some expensive
    ISL operations may take. Set to 0 to disable. An operation is canceled, if
    the limit is exceeded (e.g. the generating of a schedule is discontinued.).
  \item[barvinokBinary] binary that permits access to \textsc{libbarvinok}.
  \item[barvinokLibraryPath] Barvinok library path
  \item[paramValMappings] Value mappings for context parameters (n=42,m=43)
  \item[scheduleEquivalenceRelation] Two schedules can be see as equal according
    to different criteria. Polyite uses the following that are reasonably fast
    to determine in a search based optimization:
    \begin{itemize}
      \item RATIONAL\_MATRIX\_AND\_GENERATORS: From the sampling of the schedules
        results a schedule coefficient matrix with rational coefficients. With
        this option two schedules will be considered equivalent, if they have
        the same rational coefficient matrix.
      \item INT\_MATRIX: A schedule coefficient matrix with rational
        coefficients will be transformed into a coefficient matrix with integer
        coefficients before turning it into a schedule tree. With this option
        two schedules will  be seen considered, if they have the same integer
        coefficient matrix.
      \item SCHED\_TREE: With this option two schedules will be considered
        equivalent if they correspond to the same simplified schedule tree.
    \end{itemize}
\end{description}

\section{SCoP}

\begin{description}
  \item[benchmarkName] Name of the benchmark. Gets passed to the benchmarking
    script.
  \item[functionName] Name of the kernel function. Gets passed to the
    benchmarking script.
  \item[scopRegionStart, scopRegionEnd] Together form the name of the
    \texttt{LLVM} IR code region to be optimized. Gets passed to the
    benchmarking script.
  \item[irFilesLocation] Tells the benchmarking script where to find the source
    code (could be C code or \texttt{LLVM} IR, depending on how one uses LLVM).
  \item[seed] Random seed for \textsc{Polyite} (Not reasonable, if
    $\texttt{numScheduleGenThreads} > 1$). May be either an integer or NONE.
\end{description}

\section{Sampling}

\begin{description}
  \item[numScheds] total number of schedules to sample.
  \item[numScheduleGenThreads] The number of threads that generate schedules in
    parallel.
  \item[samplingStrategy] Strategy to sample schedules from search space
    regions. The value must be one of
  \begin{itemize}
    \item CHERNIKOVA: Sampling technique that is based on the decomposition
      theorem for polyhedra and Chernikova's Algorithm
    \item PROJECTION: Sampling technique that is based on Fourier-Motzkin
      elimination. Projection Sampling is more efficient than Chernikova
      Sampling, but preserves no information regarding the search space region
      from which a schedule matrix has been sampled.
    \item GEOMETRIC\_DIVIDE\_AND\_CONQUER: Given a search space region,
      geometric divide and conquer sampling can sample uniformly schedule
      matrices from that region. The algorithm has originally been described by
      Igor Pak (\enquote{On Sampling Integer Points in Polyhedra},
      \textit{Foundations of Computational Mathematics}, 2002) and has been
      optimized by Danner (\enquote{A performance prediction function based on
      the exploration of a schedule search space in the polyhedron model.},
      Master's thesis, 2017). Due to its many invocations of Chernikova's
      algorithm, the sampling strategy is inefficient.
  \end{itemize}
  \item[probabilityToCarryDep] During sampling of search space regions, the
    probability that all schedule coefficient vectors in a dimension polyhedron
    $P_d$ carry at least one dependence.
  \item[maxNumSchedsAtOnce] Maximum number of schedules to sample from the
    currently selected search space regions
  \item[randSchedsTimeout] Timeout in seconds for generating one random
    schedule.
  \item[genSchedsMaxAllowedConseqFailures] Sampling of schedules fails after
    so many consecutive failures to generate another schedule.
  \item[completeSchedules] If set to true, \textsc{Polyite} makes sure that a
    schedule explicitly encodes every loop from the generated code.
  \item[linIndepVectsDoNotFixDims] Tweak to speed up schedule completion
    (true/ false)
\end{description}

\subsection{Configuration Parameters for Chernikova Sampling}
\begin{description}
  \item[rayCoeffsRange] a value $b$ with $[1, b]$ being the value range for
    ray coefficients in schedule coefficient vector sampling.
  \item[lineCoeffsRange] a value $b$ with $[-b, b] \setminus {0}$ being the
    value range for line coefficients in schedule coefficient vector sampling.
  \item[maxNumRays] Maximum number of rays in the linear combination that
    forms one schedule coefficient vector must be one of
    \begin{itemize}
      \item $n$: use at most $n$ rays
      \item ALL: Use arbitrarily many rays
      \item RAND: When starting to sample a schedule matrix, randomly choose
        an upper bound for the number of rays.
    \end{itemize}
  \item[maxNumLines] Maximum number of lines in the linear combination that
    forms one schedule coefficient vector must be one of
    \begin{itemize}
      \item $n$: use at most $n$ lines
      \item ALL: Use arbitrarily many lines
      \item RAND: When starting to sample a schedule matrix, randomly choose
        an upper bound for the number of lines.
    \end{itemize}
  \item[moveVertices] When modeling a search space region, move vertices with
    rational coordinates to neighboring vertices with integer coordinates that
    are inside the polyhedron.
  \item[rayPruningThreshold] Threshold to prune rays and lines that
    coordinates exceeding rayPruningThreshold (rational number (x/y) or NONE)
\end{description}

\subsection{Configuration Parameters for Projection Sampling}
\begin{description}
  \item[schedCoeffsMin] Minimum value of schedule coefficients.
  \item[schedCoeffsMax] Maximum value of schedule coefficients.
  \item[schedCoeffsExpectationValue] The the differences between schedule
    coefficients and $min \lbrace |\textsf{schedCoeffsMin}|,$
    $|\textsf{schedCoeffsMax}| \rbrace$ are geometrically distributed. This
    parameter specifies the distribution's expected value.
\end{description}

\subsection{Configuration Parameters for Geometric Divide and Conquer Sampling}
\begin{description}
  \item[schedCoeffsAbsMax] Maximum absolute value for schedule coefficients.
\end{description}

\section{Schedule Tree Transformation and Simplification}
\begin{description}
  \item[simplifySchedTrees] Enables schedule tree simplification (true/ false)
  \item[splitLoopBodies] Enables canonical representation of loop fusion (true/
    false)
  \item[insertSetNodes] Insert set nodes where they are known to be legal (ISL
    is not ready for this.)

  \item the following options enable the different schedule simplification
    steps (true/ false)
  \begin{itemize}
  \item \textbf{schedTreeSimplRebuildDimScheds}
  \item \textbf{schedTreeSimplRemoveCommonOffset}
  \item \textbf{schedTreeSimplDivideCoeffsByGCD}
  \item \textbf{schedTreeSimplElimSuperfluousSubTrees}
  \item \textbf{schedTreeSimplElimSuperfluousDimNodes}
\end{itemize}
  \item[numSchedTreeSimplDurationMeasurements] Number of measurements of the
    duration of schedule tree simplification. (may be NONE)
\end{description}

\section{Genetic Algorithm}
\begin{description}
  \item[probabilityToMutateSchedRow] Actually the share of schedule
    dimensions that is altered by a mutation. (value between 0 and 1).
  \item[probabilityToMutateGeneratorCoeff] Actually the share of
    generator coefficients that are mutated by generator coefficient
    replacement.

  \item[generatorCoeffMaxDenominator] Maximum denominator of rational
    generator coefficient that are introduced by generator coefficient
    replacement.
  \item[currentGeneration] Index of the generation, from which the
    genetic algorithm should start. If the value is not zero, the presence of
    a file \texttt{<populationFilePrefix><index>.json} is expected, from which
    the schedules of the current generation can be loaded.

  \item[maxGenerationToReach] The index of the maximum generation to
    reach.
  \item[regularPopulationSize] Size of a single population.
  \item[fractionOfSchedules2Keep] Fraction of schedules that survive
    a generation and can reproduce. Format: (n/d)
  \item[maxNumNewSchedsFromCrossover] More than one schedule can
    result from one geometric crossover. This parameter controls the maximum
    number.
  \item[optimizeParallelExecTime] Controls whether to optimize
    sequential or parallel execution time.
  \item[evolutionTimeout] A timeout in seconds for mutations and
    crossovers. This is necessary, since many operations on polyhedra have at
    least exponential execution time in the worst case.
  \item[shareOfRandSchedsInPopulation] Share of randomly generated
    schedules in each generation. Set this to a small rational number. Format:
    (n/d)

  \item[replaceDimsEnabled] Enable mutation by dimension replacement
    (true/false)
  \item[replacePrefixEnabled] Enable mutation by schedule prefix
    replacement (true/false)
  \item[replaceSuffixEnabled] Enable mutation by schedule suffix
    replacement (true/false)
  \item[mutateGeneratorCoeffsEnabled] Enable mutation by generator
    coefficient replacement (true/false). Must only be used in combination
    with Chernikova Sampling.
  \item[geometricCrossoverEnabled] Enable geometric crossover
    (true/false)
  \item[rowCrossoverEnabled] Enable row crossover (true/false)
  \item[initPopulationNumRays] Works like \textbf{maxNumRays} but for
    the generation of the initial population.
  \item[initPopulationNumLines] Works like \textbf{maxNumLines} but
    for the generation of the initial population.
  \item[useConvexAnnealingFunction] Switch between convex and concave
    annealing function. Due to a bug, convex is concave (true/false).
  \item[gaCpuTerminationCriterion] If \textbf{evaluationStrategy} = CPU or
  \textbf{evaluationStrategy} = CLASSIFIER\_AND\_CPU the genetic algorithm's
  termination criterion is configurable. The following two options are
  available:
  \begin{itemize}
    \item FIXED\_NUM\_GENERATIONS: The genetic algorithm terminates after
      the evaluation of generation \textsf{maxGenerationToReach}. The initial
      generation has index 0.
    \item CONVERGENCE: This termination criterion is parameterized by the
      following configuration options:
      \begin{description}
        \item[convergenceTerminationCriterionWindowSize] The length of a sliding
          window that contains the most recent generations. The window's minimum
          length is 2.
        \item[convergenceTerminationCriterionThreshold] A value in $[0, 1]$
          that serves as a threshold for the relative difference in speedup
          yielded between the best schedules in two subsequent generations.
      \end{description}
        The genetic algorithm will terminate, if within the window of the latest
        generations the improvement in maximum speedup did not exceed the
        configured threshold for any pair of subsequent generations.
  \end{itemize}
  If the schedule evaluation relies on the classifier, the genetic algorithm
  will terminate if the number of generations exceeds
  \textsf{maxGenerationToReach} or if the share of profitable schedules in the
  population exceeds 95\%.
\end{description}

\subsection{Distributed Genetic Algorithm}
Polyite's genetic algorithm can run in a single process or distributed across
multiple processes that communicate using OpenMPI. In single process mode
thread-level parallelism is available to parallelize schedule generation,
crossover, mutation, and evaluation. In distributed mode, we have multiple
processes that each are responsible for a subset of the genetic algorithm's
population. Frequently, schedules migrate between the subpopulations via
OpenMPI.
\begin{description}
  \item[executionMode] Specifies whether to run in single process mode or in
    distributed mode. Allowed values are MPI and SINGLE\_PROCESS.
  \item[migrationStrategy] Sets the migration strategy to be used for the
    distributed genetic algorithm. Allowed values are NEIGHBOR and NEIGHBORHOOD.
  \item[migrationRate] The frequency in number of generations at which
    schedules will be migrated between the processes. Must be a positive
    integer.
  \item[migrationVolume] The volume of schedules to be migrated. The relative
    share of schedules to replace by migration must be specified as a rational
    number in $(0, 1]$ (Format: (n/d).
\end{description}

\section{Random exploration with a Search Space Construction Similar to LeTSeE}
\begin{description}
  \item[boundSchedCoeffs] Enable or disable the limitation of the
    schedule coefficients to $\lbrace -1, 0, 1 \rbrace$. (true/false)
  \item[completeSchedules] Allows to choose whether to append
    dimensions to the sampled coefficient matrices to encode all loops
    explicitly, or not. (true/false)
  \item[depsWeighingMethod] Besides the method to weigh dependence polyhedra by
    the dependent statements' approximate memory traffic
    (APPROX\_STMT\_MEM\_TRAFFIC), we allow for the calculation of the
    statements' exact memory traffic (STMT\_MEM\_TRAFFIC) or an approximation
    of the volume of data exchanged between the dependent statements' memory
    accesses (DEP\_CACHE\_FOOT\_PRINT).
\end{description}

\section{Evaluation}
\begin{description}
  \item[evaluateScheds] Enables schedule evaluation.
  \item[numMeasurementThreads] Number of threads for parallel schedule
    evaluation.
  \item[tilingPermitInnerSeq] Tell \textsc{Polyite} that the code optimizer
    will tile inner fused loops that are represented in the schedule tree by a
    band node followed by sequence node whose children are leaf nodes. Polly
    does so since version~5.0.
    \item[evaluationStrategy] Strategy for the evaluation of schedules'
    fitness. The value must be one of the following:
    \begin{itemize}
      \item CPU: Determine schedules' fitness by generating code and measuring
      the code's execution time. Also the computation result can be verified and
      any schedules that yield invalid computation results will be discarded. A
      schedule that causes the compiler to fail will also be discarded.
      Schedules that the compiler considers mathematically illegal must not
      occur and will cause an immediate failure of Polyite.
      \item CLASSIFIER: The classifier relies on a machine learned surrogate
      performance model and distinguished likely profitable from likely
      unprofitable schedules.
      \item CLASSIFIER\_AND\_CPU: Use the classifier as a guard for
      benchmarking: Benchmarking will only evaluate schedules that the
      classifier considers profitable.
    \end{itemize}
\end{description}

\subsection{Benchmarking}
\begin{description}
  \item[measurementCommand] command to run the measurement script. Must run
    synchronously and permit communication via STDIN/ STDOUT.
  \item[numactlConf] \textsc{NUMACTL} config that is passed to the measurement
    command.
  \item[evaluationSigIntExitCode] The measurement should terminate with this
    exit code, when interrupted or terminated from outside.
  \item[measurementTimeout] Timeout for schedule evaluation in seconds.
  \item[compilationTimeout] Timeout for compilation in seconds. Should be $<=$
    measurementTimeout (can be set to NONE)
  \item[measurementWorkingDir] Working of the measurement command.
  \item[measurementTmpDirBase] base directory for temporary working directories.
  \item[measurementTmpDirNamePrefix] Name prefix of the directory that
    measurement processes can create and use to store temporary data.
  \item[referenceOutputFile] Reference output to validate the result of
    transformed programs.
  \item[numExecutionTimeMeasurements] Number of execution time measurements
    (the minimal measured time is taken as the fitness of a schedule.)
  \item[numCompilatonDurationMeasurements] Number of compilation duration
    measurements.
  \item[validateOutput] Validate computation results.
  \item[measureParExecTime] enables parallel execution time measurement (true/
    false)
  \item[measureSeqExecTime] enables sequential execution time measurement
    (true/ false)
  \item[seqPollyOptFlags] Polly flags for sequential code generation
  \item[parPollyOptFlags] Polly flags for parallel code generation
  \item[measureCacheHitRatePar] Measure the cache hit rate (using PAPI) of
    parallel execution
  \item[measureCacheHitRateSeq] Measure the cache hit rate (using PAPI) of
    sequential execution
  \item[benchmarkingSurrenderTimeout] If \textsc{Polyite} fails to start a
    benchmarking process, it will retry every 15 minutes but fails after the
    given timeout (in seconds). May be NONE.
\end{description}

\subsection{Classification}
\begin{description}
  \item[learningSet] A comma separated list of CSV files that have been produced
    by the \texttt{fitness\_calculator} utility program.
  \item[normalizeFeatures] Should values of schedule features be normalized to
    the interval $[0, 1]$? Make sure that the data in the learning sets reflects
    this choice. (true / false)
  \item[learningAlgorithm] Choose the machine learning technique to learn the
  classifier. Possible choices are
  \begin{itemize}
    \item CART: Learn a single decision tree from the training data.
    \item RANDOM\_FOREST: Learn a randomized forest of decision trees. A
      prediction will be the prediction by the majority of the trees in the
      forest.
  \end{itemize}
  \item[decTreeMinSamplesLeaf] The algorithm that learns a classification tree
    recursively splits a training set. For each split it determines an optimal
    splitting criterion. Later, decisions will be made according to these
    criteria. The recursion stops as soon as the training set's size is smaller
    than the given positive integer value.
  \item[pythonVEnvLocation] Polyite relies on scikit-learn
    \footnote{\url{http://scikit-learn.org/}} for machine learning. scikit-learn
    runs in an interactive Python 3 session. Optionally, scikit-learn and its
    dependencies can be installed in a Python 3 virtual environment. This
    parameter must specify the path to the virtual environments root directory.
    Polyite will load the virtual environment by itself. Therefore, Polyite
    must not be started within the virtual environment. May be set to NONE.
\end{description}

\subsubsection{Additional Parameters for Random Forest}

\begin{description}
  \item[randForestNTree] The number of trees in the random forest.
  \item[randForestMaxFeatures] Besides learning each tree from a subset of the
    training data, it is also possible to consider only a subset of the features
    for the fitting of each tree. The value must be between 1 and the number of
    the features.
\end{description}

\section{Data Export /Import}
\begin{description}
  \item[populationFilePrefix] Name prefix for the population JSON file
  \item[importScheds] Import schedules from existing JSON file (the name is
    deduced from populationFilePrefix) (true/ false)
  \item[filterImportedPopulation] Do not reevaluate imported schedules where
    the evaluation failed previously.

  \item[exportSchedulesToJSCOPFiles] Export each schedule to a separate
    extended JSCOP file. (true/ false)
  \item[jscopFolderPrefix] Name prefix for the folders containing the JSCOP
    files. The name of files is generate from the imported JSCOP file's name.
  \item[exportPopulationToCSV] Export generated schedules and evaluation
    results to CSV
  \item[csvFilePrefix] Path prefix for the CSV files.
\end{description}
\end{document}
